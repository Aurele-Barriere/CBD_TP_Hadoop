Biographies

Matthieu Dorier is a postdoctoral researcher at Argonne National Laboratory (IL, USA) since February 2015. He received his Ph.D. degree in Computer Science from Ecole Normale Superieure de Rennes (France) in December 2014. His research interests include efficient I/O, data management and storage for high-performance simulations, data-intensive computing, data analytics and visualization. He has published several research papers in top HPC conferences, including SC, IPDPS and Cluster.

Gabriel Antoniu is a Senior Research Scientist at Inria, Rennes. He leads the KerData research team, focusing on storage and I/O management for Big Data processing on scalable infrastructures (clouds, HPC systems). He leads several international projects in partnership with Microsoft Research, Argonne National Lab, the University of Illinois at Urbana Champaign, IBM. He served as Program Chair for the IEEE Cluster 2014 conference and regularly serves as a PC member for major conferences in the areas of HPC, cloud computing and Big Data. He has acted for advisor of 15 PhD theses and has co-authored over 100 international publications. More: http://www.irisa.fr/kerdata/people/Gabriel.Antoniu/.

Franck Cappello

Marc Snir

Roberto Sisneros

Orcun Yildiz received his BSc degree in computer science from Bogazici University, Turkey and his MSc degree in Distributed Computing from Royal Institute of Technology, Sweden. He is currently a PhD student in INRIA, Rennes - Bretagne Atlantique Research Centre. His research interests include distributed systems, high performance computing and energy efficient big data management.

Shadi Ibrahim is a permanent Inria Research Scientist within the KerData research team. He obtained his Ph.D. in Computer Science from Huazhong University of Science and Technology in Wuhan of China in 2011. From November 2011 to September 2013, he was a postdoc researcher within the KerData research team. His research interests are in cloud computing, big data management, data-intensive computing, virtualization technology, and file and storage systems. He has published several research papers in recognized Big Data and cloud computing research conferences and journals including TPDS, FGCS, PPNA, SC, IPDPS, Mascots, CCGrid, ICPP, SCC, Cluster, and Cloudcom.

Tom Peterka is an assistant computer scientist at Argonne National Laboratory, fellow at the Computation Institute of the University of Chicago, adjunct assistant professor at the University of Illinois at Chicago, and fellow at the Northwestern Argonne Institute for Science and Engineering. His research interests are in large-scale parallelism for in situ analysis of scientific data. His work has led to two best paper awards and publications in ACM SIGGRAPH, IEEE VR, IEEE TVCG, and ACM/IEEE SC, among others. Peterka received his Ph.D. in computer science from the University of Illinois at Chicago, and he currently works actively in several DOE- and NSF-funded projects.

Dr. Leigh Orf is a Professor of Atmospheric Science and Chair of the Department of Earth and Atmospheric Sciences at Central Michigan University. Leigh received his PhD in Atmospheric Science in 1997 from the University of Wisconsin-Madison. His research focuses on thunderstorms, specifically the types that produce damaging tornadoes and downbursts. Dr. Orf’s specialty is numerical modeling, analysis, and visualization of thunderstorms on High Performance Computing (HPC) resources. Recent research includes the analysis and visualization of a breakthrough simulation of a long-track EF5 tornado embedded within a supercell thunderstorm, the first simulation of its kind.



